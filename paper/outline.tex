\documentclass[11pt]{article}

\usepackage{geometry}
\usepackage{natbib}
\bibpunct[:]{(}{)}{,}{a}{}{;}

%--------------------
%\usepackage{gb4e}
%\noautomath

\usepackage{amsmath}
\usepackage{amsfonts}
\usepackage{amsthm}
\usepackage{amssymb}
\usepackage{mathrsfs}
\usepackage{nicefrac}
%\usepackage{stmaryrd}
%\usepackage{multicol}
\usepackage{graphicx}
\usepackage{color}
\usepackage{booktabs}

\newcommand{\citeposs}[2][]{\citeauthor{#2}'s (\citeyear[#1]{#2})}
\newcommand{\tuple}[1]{\ensuremath{\left\langle #1 \right\rangle}} 

\newcommand{\hl}[1]{\textcolor[rgb]{.8,.33,.0}{#1}}% prints in orange
%\newcommand{\argmax}[1]{\underset{#1}{\operatorname{arg}\,\operatorname{max}}\;}
%\newcommand{\argmin}[1]{\underset{#1}{\operatorname{arg}\,\operatorname{min}}\;}
%\newcommand{\sbna}{\exists\lnot\forall}

\definecolor{Red}{RGB}{178,34,34}
\newcommand{\mf}[1]{\textcolor{Red}{[MF: #1]}} 
\newcommand{\tb}[1]{\textcolor[rgb]{.8,.33,.0}{[TB: #1]}}% prints in orange

\usepackage{blkarray}
\usepackage{xspace}

%%% MF's commands
\newcommand{\set}[1]{\left\{#1\right\}}
\newcommand{\card}[1]{\left \lvert \, #1 \, \right\rvert}
\newcommand{\abs}[1]{\lvert #1 \rvert}
\newcommand{\States}{\ensuremath{S}\xspace}		% Set of States
\newcommand{\state}{\ensuremath{s}\xspace}		% single states
\newcommand{\mystate}[1]{\ensuremath{\state_{\text{#1}}}\xspace} %meaningful states
\newcommand{\Messgs}{\ensuremath{M}\xspace}		% Set of Messages
\newcommand{\messg}{\ensuremath{m}\xspace}		% single messages
\newcommand{\mymessg}[1]{\ensuremath{\messg_{\text{#1}}}\xspace} %meaningful messages
\newcommand{\ssome}{\mystate{\ensuremath{\exists\neg\forall}}}
\newcommand{\sall}{\mystate{\ensuremath{\forall}}}
\newcommand{\msome}{\mymessg{some}}
\newcommand{\mall}{\mymessg{all}}
\newcommand{\asome}{\myact{\ensuremath{\exists\neg\forall}}}
\newcommand{\aall}{\myact{\ensuremath{\forall}}}
\definecolor{mygray}{cmyk}{0.35,0.35,0.35,0.35}
\newcommand{\mygray}[1]{{\textcolor{mygray}{#1}}}
%%% 


%--------------------
%\usepackage{gb4e}
%
%\usepackage{setspace}
%\onehalfspacing
%
%\usepackage{soul} %underlining
%\setuldepth{a}




%\usepackage{inconsolata}

%-------------------


\title{Sketch:\\Transmission perturbations in the cultural evolution of language}

\author{%\bf NAME1 and NAME2\\
    \today
}


\date{}

\begin{document}
\maketitle

\section{Introduction}
Language emergence and change are shaped by their use and transmission across generations. The latter process is often viewed as arising from a combination of general learning mechanisms and inductive cognitive biases (e.g. \citealt{griffiths+kalish:2007,kirby+etal:2014,tamariz+kirby:2016}).  Proposals of biases that shape language acquisition abound. Some prominent examples are mutual exclusivity \citep{merriman+bowman:1989,clark:2009}, a simplicity \citep{kirby+etal:2015}, regularization \citep{hudson+etal:2005}, and generalization biases \citep{smith:2011,oconnor:2015}.\footnote{Depending on their formulation and the domain(s) they are proposed to apply to, some biases entail others. For instance, a domain-independent bias for simplicity can entail regularization (and may conflict with mutual exclusivity).} Here, we show how environmental factors can produce evolutionary outcomes that look as if cognitive learning biases are present even if they are not. In doing so, we underline the pivotal role of systematic transmission perturbations of linguistic knowledge in language change while showing that such perturbations may stem from other sources. That is, they may not stem from learning biases but from other sources of systematic noise, e.g., as a result of errors in perception. This result highlights the frequently overlooked possibility that channel noise in evolutionary replication can mimic effects of inductive biases.

\begin{itemize}
  \item Brief overview of the cases we consider
\end{itemize}


\section{Model}

\begin{itemize}
  \item Introduction to iterated Bayesian learning. Highlighting predictions of the prior's influence (i) from a technical perspective with respect to sampling to MAP and (ii) conceptual perspective with respect to the claim that laboratory experiments can give us insights into learning priors \tb{For this we can draw directly from what we already had in the other draft}
  \item Noisy iterated learning. Following what we had before.
  \item Possibly: functional pressure \tb{Required for vagueness.}
\end{itemize}

We denote the probability that the teacher (learner) observes state $s_t$ ($s_l$) when the actual state is $s_a$ as $P_N(s_t \mid s_a)$ ($P_N(s_l \mid s_a)$). The probability that $s_a$ is the actual state when the learner observes $s_l$ is therefore:

\begin{align*}
  P_N(s_a \mid s_l) \propto P(s_a) \ P_N(s_l \mid s_a)\,.
\end{align*}

Accordingly, the probability that the teacher observes $s_t$ when the learner observes $s_l$ is:
\begin{align*}
  P_N(s_t \mid s_l) = \sum_{s_a} P(s_a \mid s_l) \ P_N(s_t \mid s_a)\,.
\end{align*}
Finally, this gives us the probability that a teacher of type $t$ produces a datum that is
perceived by the listener as $d = \tuple{s_l, m}$:
\begin{align*}
  P_N(\tuple{s_l, m} \mid t) = \sum_{s_t} P_N(s_t \mid s_l) \ P(m \mid s_t; t)\,.
\end{align*}
Generalize this to a sequence of perceived data $d_l$ and write $P_N(d_l \mid t)$. Then, the noise-perturbed mutation matrix is defined as:
\begin{align*}
  Q_{ij}  \propto \sum_{d_l \in D} P(d_l \mid t_i) F(t_j,d_l) \,, \ \  \text{where $F(t_j,d)$
    is as before.}
\end{align*}
In words, it may be the case that learner and/or teacher do not perceive the actual state as what it is. They are not aware of this, and produce/learn as if what they observed was the actual state. In particular, the learner does not reason about noise when she tries to infer the speaker's type. She takes what she observes a state to be as the actual state that the teacher has seen as well and infers which type would have most likely generated the message to this state. This can lead to biases of inferring the ``wrong'' teacher type if the noise makes some types err in a way that resembles the noiseless behavior of other types. That is, such environmental factors can, in principle, induce transmission biases that look as if there was a cognitive bias in favor of a particular type, simply because that type better explains the noise.

\section{Applications}
\tb{If we only look at iterated learning, we can skip receiver strategies. In any case, it may be best to assume that a single type's strategy is derived from a single language in order to avoid having too many types.}

\subsection{Vagueness setup}
\tb{To be specified}
\section{Deflation}
\begin{itemize}
  \item $S = [0,99]$
  \item $|M| = 1$
  \item There is one type of signaling behavior per threshold $\theta$ and one threshold per state, i.e. $100$ types.
  \item $P(m|s,t) = 1$ iff $s \geq \theta_t$, otherwise no message is sent. \tb{I'm pretty confident that adding some error-rate to this behavior wouldn't change the predictions. I left it deterministic for the time being}
  \item $P(s_{\text{perceived}} | s_{\text{actual}})$ is the probability density of getting $s_{\text{perceived}}$ from $\text{Normal}(s_{\text{actual}},\sigma)$
  \item Data generated by teachers is sampled without noise to get a representative sample. But actual likelihoods of producing the data used to compute $Q$ are subjected to noise as above (as specified above)
  \item Learners are not aware of noise (as specified above)
  \item No replication. \tb{Not sure how this would work anyway. The higher $\theta$, the less a type communicates. If that's a communicative failure, then these types are even more dispreferred than with only learning. If it's not, then we have the same fitness for each type}
\end{itemize}

\section{Quantifiers}
\tb{To be specified} 

\section{Discussion}
\tb{To be specified. Parts can be taken from our previous draft} 

\section{Conclusion}

\bibliographystyle{unsrtnat}

%\setlength{\bibleftmargin}{.125in}
%\setlength{\bibindent}{-\bibleftmargin}
\bibliography{./noise-bib}


\end{document}
